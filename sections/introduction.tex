\section{Introduction}

Diabetes mellitus represents one of the most pressing global health challenges of the 21st century, affecting an estimated 537 million adults worldwide as of 2021, with projections suggesting this number will exceed 783 million by 2045 \cite{mackenzie2023diabetes}. The disease imposes substantial burdens on healthcare systems, accounting for approximately 12\% of global health expenditure, while contributing to significant morbidity and mortality through microvascular and macrovascular complications \cite{guan2023artificial}. Traditional approaches to diabetes management—relying on periodic clinical assessments, self-monitoring of blood glucose, and standardized treatment protocols—struggle to address the heterogeneous nature of the disease and the complex interplay of genetic, behavioral, and environmental factors that influence glycemic control \cite{oikonomou2023machine}.

The convergence of digital health technologies, ubiquitous sensing, and advances in artificial intelligence (AI) and machine learning (ML) has catalyzed a paradigm shift toward data-driven, personalized diabetes care \cite{jacobs2023artificial, wang2024ai}. Wearable devices and continuous glucose monitoring (CGM) systems generate high-resolution time-series data capturing glucose dynamics, physical activity, heart rate variability, and other physiological signals \cite{alhaddad2022sense}. Retinal imaging modalities—fundus photography and optical coherence tomography (OCT)—enable early detection of diabetic retinopathy, a leading cause of preventable blindness \cite{contreras2018artificial}. Electronic health records (EHRs) aggregate longitudinal clinical data, laboratory results, and medication histories. Integrating these multimodal data sources through AI/ML frameworks promises predictive analytics, personalized treatment recommendations, and early intervention for complications \cite{mackenzie2023diabetes, khalifa2024artificial}.

However, the translation of AI/ML research into clinically robust, equitable, and scalable diabetes care systems faces formidable challenges that extend beyond algorithmic performance on benchmark datasets. Data heterogeneity arising from diverse sensor technologies, acquisition protocols, and patient populations complicates model development and validation \cite{prioleau2025deep}. Deep learning models, while achieving state-of-the-art performance in controlled settings, often exhibit poor generalization when deployed in new clinical environments or applied to populations underrepresented in training data \cite{zhu2021deep, alam2024machine}. Algorithmic bias and fairness concerns are pervasive, with models systematically underperforming for female patients, individuals with suboptimal glycemic control, and racial and ethnic minorities \cite{prioleau2025deep, wang2024ai}. Interpretability and explainability remain critical barriers to clinical adoption, as healthcare providers require transparent reasoning to trust AI-generated recommendations \cite{jacobs2023artificial, alam2024machine}.

These challenges manifest with particular acuity in low-resource settings, where infrastructure limitations, device costs, data scarcity, and workforce constraints create additional barriers to AI deployment \cite{mackenzie2023diabetes, ghosh2025artificial}. African populations, for instance, face disproportionate diabetes burdens yet remain severely underrepresented in AI research and development \cite{bai2024federated}. The lack of diverse training data, coupled with models optimized for high-resource clinical environments, results in AI systems that may fail catastrophically when applied to contexts characterized by intermittent connectivity, limited specialist availability, and heterogeneous patient populations \cite{fahmy2025exploring}. This reality underscores the need for a critical examination of AI/ML methodologies through the lens of accessibility, affordability, and generalization across diverse real-world conditions.

Recent work has begun to address these concerns through frameworks emphasizing inclusive healthcare, integrating education and research with AI and personalized curricula \cite{bahmani2025achieving}. Such approaches recognize that democratizing AI-driven precision medicine requires not only technical innovation but also capacity building, stakeholder engagement, and explicit consideration of equity from the earliest stages of system design \cite{bai2024federated}. Federated learning paradigms, which enable collaborative model training without centralizing sensitive patient data, offer promising pathways for privacy-preserving, multi-institutional research \cite{fahmy2025exploring, bai2024federated}. Transfer learning and domain adaptation techniques may facilitate model generalization across populations and clinical settings \cite{contreras2018artificial}. Yet these methodological advances remain nascent, with limited evidence of real-world impact in resource-constrained environments.

This narrative review provides a comprehensive, critical synthesis of AI/ML challenges in multimodal diabetes data, with particular emphasis on wearable sensors, retinal imaging, and model generalization. We examine the current landscape of methodological approaches, data modalities, and fusion strategies, while highlighting persistent limitations related to bias, fairness, interpretability, and deployment. By foregrounding the experiences and constraints of low-resource settings—using African populations as a contextual lens—we aim to illuminate the gap between algorithmic promise and clinical reality, and to identify research priorities that can advance equitable, robust, and scalable AI-driven diabetes care.
