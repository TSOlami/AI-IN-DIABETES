\section{Data Modalities and Problem Formulation}

AI systems for diabetes care draw on several distinct types of data, each with its own strengths and limitations. Understanding what these data sources can and cannot provide is essential for building systems that work reliably in practice.

\subsection{Wearable Sensors and Continuous Glucose Monitoring}

Continuous glucose monitoring has transformed how patients and clinicians track blood sugar. These devices measure glucose in the fluid beneath the skin every 1 to 15 minutes, producing detailed records of how glucose levels change throughout the day \cite{alhaddad2022sense, contreras2018artificial}. This continuous stream enables detection of dangerous highs and lows, prediction of future glucose trends, and fine-tuning of insulin doses \cite{jacobs2023artificial}. Beyond CGM, smartwatches and fitness trackers now capture heart rate, physical activity, and sleep patterns, all of which influence glucose regulation \cite{alhaddad2022sense, rodriguez2023applications}.

Combining CGM with activity data creates opportunities for smarter predictions. When a model knows that someone just exercised or had a poor night's sleep, it can adjust its glucose forecast accordingly \cite{contreras2018artificial, alhaddad2022sense}. Researchers have also explored whether heart rate signals from smartwatches could estimate glucose without any needle, though these methods have not yet proven accurate enough for clinical use \cite{zanelli2022diabetes, alhaddad2022sense}.

Despite their promise, wearable data create real headaches for AI developers. Different devices sample at different rates and use different measurement techniques, making it hard to build models that work across brands \cite{prioleau2025deep, langarica2024deep}. Sensors fall off, batteries die, and patients forget to wear them, leading to gaps in the data that can distort model training and predictions \cite{xie2024processing, hirsch2024interruption}. Calibration drift means the same sensor may give slightly different readings over time \cite{alhaddad2022sense, jacobs2023artificial}. Processing these high-frequency, noisy signals requires specialized neural network architectures like LSTMs, which need substantial computing power and large training datasets to work well \cite{zhu2021deep, alam2024machine, martinsson2020blood}.

\subsection{Retinal Imaging Modalities}

Diabetic retinopathy remains one of the most serious complications of diabetes, affecting about 35\% of patients and ranking among the leading causes of blindness in working-age adults \cite{yau2012global}. Catching it early through routine screening can prevent vision loss, but many patients never receive the eye exams they need. Fundus photography and optical coherence tomography (OCT) are the two main imaging techniques used for screening and diagnosis \cite{contreras2018artificial}.

AI-based retinal screening has achieved remarkable success in controlled settings. Fundus cameras capture color images of the back of the eye, revealing the tiny bleeds, protein deposits, and abnormal blood vessels that signal retinopathy \cite{selvachandran2022developments, abushawish2024deep}. A systematic review of regulator-approved deep learning systems found pooled sensitivity of 93\% and specificity of 90\%, performance that matches or exceeds trained human graders \cite{zhang2025systematic, liu2023prospective, wu2023efficacy}. OCT adds three-dimensional detail by imaging the retinal layers, helping clinicians assess swelling and structural damage \cite{contreras2018artificial, padhy2024systematic}.

Yet translating these results into real-world screening programs has proven difficult. Image quality varies enormously depending on the camera, the operator, and patient factors like cataracts or pupil dilation. A meta-analysis found that 3-12\% of screening images are ungradable, with rates higher for non-mydriatic photography (12.1\%) than mydriatic approaches (3.4\%) \cite{zhang2025systematic, kanclerz2021imaging}. Models trained on research-quality images often struggle with the lower-quality photos from portable cameras used in community settings \cite{raj2024federated, zhang2025systematic}. Most training datasets come from high-income countries, raising concerns about performance in underrepresented populations \cite{wang2024ai, olusanya2024mitigating}. Even the "ground truth" labels used for training contain noise, since human graders often disagree about borderline cases \cite{zhang2025systematic}.

\subsection{Electronic Health Records and Clinical Data}

Electronic health records contain years of clinical history: diagnoses, lab results, medications, vital signs, and physician notes \cite{contreras2018artificial, oikonomou2023machine}. Mining these records allows researchers to identify which patients face the highest risks and might benefit from earlier intervention \cite{wang2024ai}. Natural language processing can extract structured information from free-text notes, unlocking data that would otherwise require manual review \cite{contreras2018artificial}.

Working with EHR data presents its own challenges. Records are messy, with thousands of potential variables, many of them missing or recorded at irregular intervals \cite{oikonomou2023machine}. The patients with the most complete records tend to be those who visit doctors most often, which may bias models toward sicker populations \cite{jacobs2023artificial}. The complex interactions between medications, other health conditions, and lifestyle factors make traditional statistical methods insufficient, pushing researchers toward deep learning approaches that can capture nonlinear patterns \cite{zhu2021deep}.

\subsection{Problem Formulation and Modeling Objectives}

Researchers apply AI to diabetes care for several distinct purposes, each requiring different data, methods, and success metrics:

\begin{itemize}
    \item \textbf{Glucose prediction:} Forecasting where blood sugar will be 30 to 60 minutes ahead, giving patients time to eat or take insulin before problems develop \cite{jacobs2023artificial, prioleau2025deep}.
    \item \textbf{Retinopathy screening:} Automatically grading eye images to determine which patients need specialist referral, enabling screening at scale \cite{zhang2025systematic, zhu2021deep}.
    \item \textbf{Risk prediction:} Estimating the likelihood that someone will develop diabetes or experience a complication like kidney disease or amputation \cite{oikonomou2023machine, wang2024ai}.
    \item \textbf{Treatment optimization:} Recommending personalized insulin doses, medication adjustments, or lifestyle changes \cite{mackenzie2023diabetes, contreras2018artificial}.
\end{itemize}

Each task requires careful thought about how to define outcomes, how far ahead to predict, and what metrics matter clinically \cite{jacobs2023artificial}. Rare but serious complications create imbalanced datasets where standard accuracy measures can be misleading \cite{alam2024machine}. The progressive nature of diabetes means that models must account for how patterns change over time, adding complexity that simpler approaches cannot handle \cite{zhu2021deep}.

Table~\ref{tab:data_modalities} summarizes the key data types, their clinical uses, and the challenges each presents.

\begin{table*}[t]
\centering
\caption{Comparison of Data Modalities in AI-Driven Diabetes Care}
\label{tab:data_modalities}
\begin{tabular}{p{2.5cm}p{3cm}p{4cm}p{5.5cm}}
\toprule
\textbf{Data Modality} & \textbf{Key Characteristics} & \textbf{Clinical Applications} & \textbf{Major Challenges} \\
\midrule
Continuous Glucose Monitoring (CGM) & High-frequency time-series (1-15 min intervals); interstitial glucose & Glucose prediction, hypoglycemia detection, insulin optimization & Missing data, sensor noise, calibration drift, inter-device variability, high computational cost \\
\midrule
Wearable Sensors (Activity, HR, PPG) & Multi-modal physiological signals; variable sampling rates & Context-aware glucose prediction, activity detection, stress monitoring & Heterogeneous devices, missing data, limited accuracy for glucose estimation, privacy concerns \\
\midrule
Fundus Photography & 2D color retinal images; variable quality & Diabetic retinopathy screening and grading & Image quality variability, limited diversity in training data, device heterogeneity, label noise \\
\midrule
Optical Coherence Tomography (OCT) & High-resolution 3D retinal structure & Macular edema quantification, structural assessment & High cost, limited availability in low-resource settings, computational complexity \\
\midrule
Electronic Health Records (EHR) & Longitudinal clinical data; structured and unstructured & Risk prediction, complication forecasting, treatment optimization & High dimensionality, sparsity, irregular sampling, non-random missingness, documentation bias \\
\midrule
Genomic Data & Genetic variants, polygenic risk scores & Risk stratification, precision medicine & High cost, limited clinical utility, ethical concerns, population stratification \\
\bottomrule
\end{tabular}
\end{table*}
