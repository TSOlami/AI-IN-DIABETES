\textbf{Background:} Diabetes mellitus affects over 500 million people globally. Artificial intelligence (AI) and machine learning (ML) offer transformative potential for prevention, diagnosis, and management. However, integrating multimodal data sources—wearable sensors, continuous glucose monitoring (CGM), and retinal imaging—presents substantial methodological and system-level challenges inadequately addressed in current literature.

\textbf{Objective:} This narrative review critically examines AI/ML challenges in multimodal diabetes care, focusing on data heterogeneity, model generalization, algorithmic bias, interpretability, and deployment constraints in resource-limited settings. We emphasize how these challenges manifest acutely in underserved populations, using African contexts as a stress test for AI system robustness.

\textbf{Methods:} We conducted a comprehensive literature review of 361 unique papers published between 2018 and 2025, drawing from SciSpace, Google Scholar, ArXiv, PubMed, and npj Digital Medicine. We analyzed methodological approaches, data modalities, fusion strategies, and reported limitations across wearable time-series data, retinal imaging, and multimodal integration frameworks.

\textbf{Results:} Current AI/ML approaches demonstrate promising performance on benchmark datasets but exhibit critical weaknesses in generalization across populations, clinical settings, and data acquisition protocols. Deep learning architectures (CNNs, RNNs, LSTMs, Transformers) dominate recent literature, yet reproducibility remains poor due to dataset homogeneity, limited external validation, and inadequate reporting standards. Multimodal fusion strategies show potential for improved diagnostic accuracy but face challenges in handling missing modalities, temporal misalignment, and computational complexity. Bias and fairness issues are pervasive, with models systematically underperforming for female patients, individuals with poor glycemic control, and populations underrepresented in training data. Deployment in low-resource settings is hindered by infrastructure limitations, device costs, data privacy concerns, and lack of clinical validation in diverse populations.

\textbf{Conclusions:} Achieving equitable, robust AI-driven diabetes care requires fundamental shifts in data collection practices, model development paradigms, and evaluation frameworks. Future research must prioritize external validation across diverse populations, standardized reporting of demographic characteristics and performance disparities, federated learning approaches for privacy-preserving collaboration, and explicit consideration of cost, infrastructure, and accessibility constraints. Without addressing these challenges, AI systems risk exacerbating existing health inequalities rather than democratizing precision medicine.
